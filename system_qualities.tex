\chapter{System qualities}
\label{cha:system-qualities}
\thispagestyle{fancy}


\section{Performance and scalability}
\label{sec:perf-scal}
The main performance and scalability requirements for ORS are Q1, Q3 and Q4 (see Section \ref{sec:syst-qual-scen}).

Scaleability, corresponding to Q1, is not a problem as we are using Amazon Cloud (AWS); it can scale to well over 300.000 concurrent users at peak time, see \cite{amazonCaseStudy}. Thus scalability of ORS is dependent on our own developed software, which we are unable to test.

Amazon Cloud (AWS) has a guaranteed uptime, corresponding to Q4, of $99.95\%$, see \cite{amazonSla}. Thus we need not worry about disc failures and other hardware problems, making the software (both the self developed system and the backend systems like the databases) the most significant weak point of availability of ORS.

Regarding Paypal, the only data we could find was based on December 2011, see \cite{pingdom}. The uptime was $99.998\%$ but the response time of Paypal was very bad with 3.800 ms. According to \cite{basicState}, Paypal had in October an uptime around $93.89\%$ but a better response time of 118 ms.

We have not been to test the stability of our own software, and are thus not able to estimate uptime or other performance related issues.

We have not been able to performance test response time of Q3, but note that this only concerns that our end of the system should respond within 0.1 seconds, not that the user should recieve a response within 0.1 seconds (which would be hard to guarantee due to network factors/constraints such as distance, speed and congestion). Thus we need to design our front-end to respond on user interaction fast.


\section{Security}
\label{sec:security}
The security of the system mainly revolves around user-data and payment. This
means that the system generally can relax its constraints on the item data,
those are still vital to the systems usability, but they are not critical or
sensitive as the before-mentioned types of data. The table
\ref{tab:security-resources} defines and describes these resources, where AC is
Access Control.

The main requirements to be satisfied is Q2 and Q5, further the consequences of
Q3 not being met is discussed loosely. The table
\ref{tab-security:requirements}, describes the requirements and how they are
fulfilled.

Attack trees has been considered, as a tool to identify threats to the system.
But the normal attack vectors against the chosen Amazon platforms has not been
publicly been explored thoroughly enough to conclude what measures to take.
Instead all code shall follow best practice for the platform and adhere to
common security practices.

\begin{table}[ht]
    \centering
    \begin{tabular}{p{3cm} | l | l | p{4cm}}
        Resource & Owner & Sensitivity & AC \\ \hline
        User-data(Passwords and address) & User & Medium & Access to creation
            and modification by authenticated and authorized users \\ \hline
        Payment(Credit card) & User(PayPal) & High & Delegated to 
            PayPal \\ \hline
        Item data(Quantity, Description, Picture) & ORS & Low & Creation and
            modification controlled by Access Control Lists \\
    \end{tabular}
    \label{tab:security-resources}
    \caption{Security resources}
\end{table}

\begin{table}[ht]
    \centering
    \begin{tabular}{l | p{5cm} | p{7cm}}
        Requirement ID & Requirement & Fulfilling \\ \hline
        Q2 & A user tries to access a page without valid credentials &
        The frontend system, consults with the user database, as to check if the
        user has valid credentials. \\ \hline
        Q5 & A user is authenticated, but not authorized to change an items
        data & The frontend system, consults with the user database, that also
        maintains Access Control Lists. This also supports having the
        administrative pages within the main system. \\ \hline
        NA & SQL Injections & The system shall as soon as possible sanitize
            strings, this could with high likelihood be done in the router. \\
    \end{tabular}
    \label{tab:security-requirements}
    \caption{Security requirements}
\end{table}

The requirement Q3, concerns availability, and needs some special treatment
when talking about security. When a system becomes unresponsive, the result
might be error pages, revealing internal details of the system, wanted by
attackers. Further if the system suddenly becomes unresponsive, in the middle
of a payment, a customers payment may not be correctly registered, and they are
therefore forced to pay twice, or contact support.

\section{Availability and resilience}
\label{sec:avail-resil}


\section{Evolution}
\label{sec:evolution}
In table \ref{tab:evolution} is listed some likely evolutions and their expected magnitude (Lov/Medium/High), likelihood of being necessary (Low/Medium/High), and the timescale on which the evolution is expected to happen (Long/Medium/sHort).

Many of the evolution types are of low likelihood, and those that are of medium or high likelihood are easy and fast to implement. The system is designed to be easily expanded in some aspects which are likely to happen. The highest priority evolution types are also those that are most likely to happen:
\begin{itemize}
\item The addition of new query types are very likely, as more features are iteratively added to the system. This includes simple extensions on existing queries like more user information, but also new queries necessary to new features, like extraction of users based on location in case of a new map feature.
\item Supporting twice the number of users. This is mostly handled in the cloud, but might include optimization of internal components.
\item Addition of new functional components for end users, e.g. a calender service, a map service, a forum, etc.
\item New external systems and their interfaces are likely when implementing new features and components, e.g. Google Map, Facebook, Twitter, etc. 
\end{itemize}

\begin{table}[ht]\label{tab:evolution}
\centering
\begin{tabular}{| p{5cm} | l | l | l | l |}
    \hline
    & Type & Magnitude & Likelihood & Timescale \\ \hline \hline
    New query type (including expansion of DB system) & Functional & L & H & M \\ \hline
    New component & Functional & M & M & M \\ \hline
    New database system & Platform & H & L & H \\ \hline
    New external interface/system & Functional & L & M & L \\ \hline
    New logging protocol & Functional & M & L & H \\ \hline
    Support twice the number of users & Growth & L & H & L \\ \hline
    New billing site & Functional & L & L & L \\ \hline
\end{tabular}
\end{table}


\section{Other qualities}
\label{sec:other-qualities}

\subsection{Accessibility}
\label{sec:accessibility}


\subsection{Internationalisation}
\label{sec:internationalisation}


\subsection{Location}
\label{sec:location}


\subsection{Regulation}
\label{sec:regulation}


\subsection{Usability}
\label{sec:usability}


